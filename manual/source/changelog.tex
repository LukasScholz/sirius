\paragraph{3.6}
\begin{itemize}
\item
   \textbf{Fragmentation tree heuristics}
\item
   \textbf{CPLEX} ILP solver support
\item
   Custom ionizations/adducts can be specified
\item
   Consider a specific list of \textbf{ionizations for Sirius}
\item
   Consider a specific list of \textbf{adducts for CSI:FingerID} (CLI only, GUI is coming soon)
\item
  \textbf{Full-featured} standalone \textbf{command line version} (headless version)
\item
  Improved \textbf{parallelization} and task management
\item
  Improved stability of CSI:FingerID webservice
\item
  Time limit for Fragmentation tree computation
\item
  Specify fields to import name and ID from .sdf into custom database (GUI).
\item
  CSI:FingerID results can be \textbf{filtered by Custom databases} (GUI).
\item
  Better filtering performance (GUI)
\item
  Bug fix in Database filtering view (GUI)
\item
  Error Reporter bug fixed (GUI)
\item
  Logging bug fixed
\item
  Many minor bug fixes
\end{itemize}


\paragraph{3.5}

\begin{itemize}

\item
  \textbf{Custom databases} can be imported by hand or via csv file. You
  can manage multiple databases within Sirius.
\item
  New \textbf{Bayesian Network scoring} for CSI:FingerID which takes
  dependencies between molecular properties into account.
\item
  \textbf{CSI:FingerID Overview} which lists results for all molecular
  formulas.
\item
  \textbf{Visualization of the predicted fingerprints}.
\item
  \textbf{ECFP fingerprints} are now also in the CSI:FingerID database
  and do no longer have to be computed on the users side.
\item
  Connection error detection and refresh feature. No restart required to apply Sirius internal proxy settings anymore.
\item
  \textbf{System wide proxy} settings are now supported.
\item
  Many minor bug fixes and small improvements of the GUI
\end{itemize}

\paragraph{3.4}

\begin{itemize}

\item
  \textbf{element prediction} using isotope pattern
\item
  CSI:FingerID now predicts~\textbf{more molecular properties}~which
  improves structure identification
\item
  improved~structure of the result output generated by the command line
  tool \textbf{to its final version}
\end{itemize}

\paragraph{3.3}

\begin{itemize}

\item
  fix missing MS2 data error
\item
  MacOSX compatible start script
\item
  add proxy settings, bug reporter, feature request
\item
  new GUI look
\end{itemize}

\paragraph{3.2}

\begin{itemize}

\item
  integration of CSI:FingerId and structure identification into SIRIUS
\item
  it is now possible to search formulas or structures in molecular
  databases
\item
  isotope pattern analysis is now rewritten and hopefully more stable
  than before
\end{itemize}

\paragraph{3.1.3}

\begin{itemize}

\item
  fix bug with penalizing molecular formulas on intrinsically charged
  mode
\item
  fix critical bug in CSV reader
\end{itemize}

\paragraph{3.1.0}

\begin{itemize}

\item
  Sirius User Interface
\item
  new output type \emph{-O sirius}. The .sirius format can be imported
  into the User Interface.
\item
  Experimental support for in-source fragmentations and adducts
\end{itemize}

\paragraph{3.0.3}

\begin{itemize}

\item
  fix crash when using GLPK solver
\end{itemize}

\paragraph{3.0.2}

\begin{itemize}

\item
  fix bug: SIRIUS uses the old scoring system by default when \emph{-p}
  parameter is not given
\item
  fix some minor bugs
\end{itemize}

\paragraph{3.0.1}

\begin{itemize}

\item
  if MS1 data is available, SIRIUS will now always use the parent peak
  from MS1 to decompose the parent ion, instead of using the peak from
  an MS/MS spectrum
\item
  fix bugs in isotope pattern selection
\item
  SIRIUS ships now with the correct version of the GLPK binary
\end{itemize}

\paragraph{3.0.0}

\begin{itemize}
\item
  release version
\end{itemize}
